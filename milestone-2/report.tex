%
%                  Politecnico di Milano
%
%         Student: Caravano Andrea
%            A.Y.: 2023/2024
%
%   Last modified: 07/09/2024
%
%     Description: Mobile Radio Networks/Wireless Networks: OpenRAN Project (Milestone 2)
%

\documentclass[a4paper,11pt]{article} % tipo di documento
\usepackage[T1]{fontenc} % codifica dei font
\usepackage[utf8]{inputenc} % lettere accentate da tastiera
\usepackage[english,italian]{babel} % lingua del documento
\usepackage{lipsum} % genera testo fittizio
\usepackage{url} % per scrivere gli indirizzi Internet e/o di riferimento nella pagina

\usepackage[hidelinks]{hyperref} % per modificare il comportamento dei collegamenti ipertestuali (+ leva colore attorno)

\usepackage[margin=0.7in]{geometry} % margine di pagina

\usepackage{graphicx} % per inserire immagini

\usepackage[outputdir=../auxil]{minted} % per colorazione automatica del codice (installare pygments da Homebrew)
% \usepackage{pythonhighlight} % per Python

\setminted{ % si può impostare il linguaggio specifico con \setminted[JSON] ad esempio
    linenos=true,
    breaklines=true,
    encoding=utf8,
    fontsize=\normalsize,
    frame=lines
}

\usepackage{fancyhdr}
\usepackage{textcomp} % per gestione intestazione e piè di pagina

\hypersetup{ % metadati di titolo e autore nel PDF
    pdftitle={Mobile Radio Networks project - A.Y. 2023/24},
    pdfauthor={Andrea Caravano}
}

\setlength{\parindent}{0pt} % rimuove l'indentazione del testo

\begin{document}
    \pagestyle{fancy}
    \fancyhead{}\fancyfoot{}
    \fancyhead[L]{\textbf{Mobile Radio Networks project}}
    \fancyhead[R]{Andrea Caravano}
    \fancyfoot[C]{\thepage}

    \title{\textbf{Mobile Radio Networks project (OpenRAN)}\\Milestone 2: Requirements and design notes}
    \author{Andrea Caravano}
    \date{Academic Year 2023--24}
    \maketitle


    \section{Project requirements}\label{sec:project-requirements}

    \subsection{Specifications}\label{subsec:specifications}

    Implement an xApp that collects PHY/MAC metrics:

    \begin{itemize}
        \item Per-UE RSRP
        \item Per-UE BER (uplink and downlink)
        \item Per-UE MCS (uplink and downlink)
        \item Cell load (i.e. allocated PRBs)
    \end{itemize}

    All values timestamped and saved in a CSV file

    500ms data collection loop

    \subsection{Meaningful values}\label{subsec:meaningful-values}

    \begin{itemize}
        \item RSRP: -44 dBm (optimal) to -144 dBm (worst)
        \item BER: 0 to 1 (or 0 to 100, in percentage)
        \item MCS: inspired by lectures exercises, 3 Mb/s (worst) to 100 Mb/s (optimal)
        \item Cell load: 24 (worst) to 275 (maximum) PRBs
    \end{itemize}


    \section{Protobuf specification}\label{sec:protobuf}

    \subsection{RAN parameters}\label{subsec:ran-parameters}

    An overall RAN parameter have been added for the cell load.

    Note that, since this is not treated as a per-UE parameter, this is added only once to the response (as expected).

    \begin{minted}{Javascript}
enum RAN_parameter{
    GNB_ID = 1;
    UE_LIST = 3;
    CELL_LOAD = 4; // added
}
    \end{minted}

    \subsection{per-UE parameters}\label{subsec:per-ue-parameters}

    RSRP, BER and MCS measurements have been added.
    BER and MCS are split in uplink and downlink directions, as per project specifications.

    \begin{minted}{Javascript}
message ue_info_m{
    // this is to identify the ue
    required int32 rnti=1;

    // specific ue's measurements (these will come from the gnb)
    optional float rsrp=2;
    optional float berul=3;
    optional float berdl=4;
    optional float mcsul=5;
    optional float mcsdl=6;
}
    \end{minted}


    \section{Snippets}\label{sec:snippets}

    \subsection{gNB message handler}\label{subsec:gnb-message-handler}

    The described Protobuf specification translates into the following code snippets, in which the most important behaviours of the application is shown.

    \begin{minted}{C++}
void ran_read(RANParameter ran_par_enum, RANParamMapEntry* map_entry){
...
case RAN_PARAMETER__CELL_LOAD:
    cell_load = (rand() % (275 - 24 + 1)) + 24;
    map_entry->value_case = RAN_PARAM_MAP_ENTRY__VALUE_STRING_VALUE;
    map_entry->string_value = int_to_charray(cell_load);
    break;
...
}
    \end{minted}

    \begin{minted}{C++}
UeListM* build_ue_list_message(){
...
ue_info_list[i]->has_rsrp = 1;
ue_info_list[i]->rsrp = -((rand() % (144 - 44 + 1)) + 44); // dBm
ue_info_list[i]->has_berul = 1;
ue_info_list[i]->berul = (double)rand() / (double)RAND_MAX; // [0, 1]
ue_info_list[i]->has_berdl = 1;
ue_info_list[i]->berdl = (double)rand() / (double)RAND_MAX;
ue_info_list[i]->has_mcsul = 1;
ue_info_list[i]->mcsul = (rand() % (100 - 3 + 1)) + 3; // Mb/s
ue_info_list[i]->has_mcsdl = 1;
ue_info_list[i]->mcsdl = (rand() % (100 - 3 + 1)) + 3;
...
}
    \end{minted}

    \subsection{xApp}\label{subsec:xapp}

    The xApp mantains the general overall behaviour, but report requests are sent every 500 ms, writing received data to a timestamped CSV file, as per specifications.
\end{document}
